\chapter*{INTRODUCTION GÉNÉRALE}
%\paragraph{}
La transition énergétique et l’utilisation des sources d'énergie renouvelables (EnR) restent un sujet d’actualité ces dernières années dans le monde. Les causes majeures sont une augmentation remarquable du prix des énergies fossiles, l’épuisement envisagé de leurs réserves dans un futur proche et l'augmentation massive des émissions de gaz à effet de serre qui ont des impacts majeurs sur le changement climatique. S'agissant des plus répandues (l'énergie solaire et l'énergie éolienne), leur développement est ralenti à cause de leur manque de  flexibilité, et de leur intermittence. D'un côté, l'essor à l’échelle mondiale des centres de prévision encourage le développement de méthodes de gestion basées sur des outils de prévision à l’aide de données météorologiques, pour prendre en compte les perturbations qui en résultent. D’un autre côté, le déploiement de systèmes multi sources multi stockages et de systèmes de gestion intelligente permet de les prendre en compte à l'échelle de micro réseaux (Smart grid).\\ Ce type de gestion basée sur un SMA (Système Multi Agents), va servir de communication entre l’extérieur du système comme le comportement du soleil et du vent pour prendre des décisions. Et la communication entre les différents flux d'énergies au sein de l'unité (composé de sources de productions, de stockage et de charge). Ce stage s’inscrit dans le cadre du développement d’une unité pour la simulation à partir des modèles mathématiques en utilisant comme outil Matlab/Simulink, pour valider ces modèles à partir des données réelles et de tests avec la plateforme OP4510 pour valider la communication entre modèles et le SMA. Ceci va être utilisé dans le projet GYSOMATE, que nous allons présenter dans la section suivante.

%\section{Part des énergies renouvelable à la Réunion}
%\section{Objectif de mon stage}
\section*{Présentation et objectifs du projet GYSOMATE}
GYSOMATE (\textbf{G}estion d\textbf{Y}namique \textbf{S}upervision et \textbf{O}ptimisation de \textbf{M}icroréseaux urbains pour l’\textbf{A}utonomie du \textbf{T}erritoire en énergie \textbf{E}lectrique) vise le développement d’un système intelligent de gestion de l’énergie. GYSOMATE repose sur le couplage d’outils de prévision de la ressource, de stratégies de commande à base de modèles et/ou de SMA et des NTIC. Ainsi, dans la (figure \ref{1.1}), on donne une représentation des caractéristiques du système physique du projet.\\
\begin{figure}[ht]
    \centering
    \includegraphics[scale=0.4]{caracteristique.png}
    \caption{Caractéristiques du système physique}
    \label{1.1}
\end{figure}
L’objet d’étude du projet est donc un îlot urbain mixte (résidentiel, tertiaire) pouvant être équipé de systèmes de recharge de véhicules électriques.
En n'intégrant pas les problématiques d'interfaçage avec le réseau électrique public, GYSOMATE vise le développement de stratégies permettant de limiter l'impact des perturbations locales sur celui-ci. Dans ce contexte, le projet prévoit la gestion de l'équilibre production-consommation avec la gestion des puissances.
\\
Le développement de l’intelligence de gestion des flux d’énergie s’appuiera sur une plateforme de simulation temps réel (OP4510), l’entrepôt de données du LE2P (CASSANDRA) et la plateforme système multi-agents élaboré par la LIM (SKUAD).\\ Porté par une équipe de 4 ingénieurs dont \textbf{Yassine GANGAT}, responsable des développements informatiques, \textbf{Nicolas COQUILLAS}, Ingénieur Hardware-In-the-Loop (HIL),\textbf{Taher ISSOUFALY}, Ingénieur Système Multi Agent, \textbf{Marie-Laure PERONY-CHARTON}, Ingénieur Valorisation/Communication, \textbf{Dominique Grondin} chef du projet et moi comme stagiaire, GYSOMATE est entré en phase opérationnelle le 2 octobre 2017 et se décompose en 4 actions
 :\\
\textbf{Action 1 : Simulateur microréseau}\\
Une plateforme de simulation temps réel (OP4510) est implantée au LE2P pour émuler des microréseaux composés d’unités de conversion d’énergie (PV), de stockage et de consommation. Cette plate-forme permettra de tester en temps réel des stratégies de gestion de l’énergie appliquées à différentes architectures de microréseau en conditions de fonctionnement réel.\\
\textbf{Action 2 : Unités de production et de stockage connectées et pilotables}\\
Des données physiques mises à disposition par les partenaires de GYSOMATE seront agrégées à l'entrepôt de données du LE2P, connecté à la plateforme de simulation OP4510. Ces unités seront également rendues pilotables par un ensemble de capteurs et actionneurs.\\
\textbf{Action 3 : Plateforme d’agrégation}\\ 
Les données énergétiques fournies par les partenaires (unités connectées) seront agrégées aux données météorologiques du LE2P.\\
\textbf{Action 4 : Système de gestion de l’énergie}\\ 
Le système de gestion de l’énergie en cours d'élaboration repose sur des SMA (Systèmes Multi Agents). Des scénarios d’équilibrage seront générés pour tester en temps réel les stratégies de gestion de l’énergie pour la supervision en temps réel du microréseau émulé.


\section*{Organisation du mémoire} Nous allons en premier lieu, parler brièvement des systèmes d'énergie hybrides. Ensuite, présenter notre travail avec des modèles effectués sous Matlab/Simulink. Au cours de cette partie, nous allons présenter deux modèles de batterie et PV sous Simulink l'une à long terme (simplifié) pour le test avec le SMA et l'autre à court terme (plus complexe) pour leur validation à partir des données réelles. Puis les deux modèles PV et Batterie seront construits en utilisant l'outil Simscape Power Systems, afin d'effectuer les simulations et analyser les résultats, avec les données disponibles du champ PV, du parc de batterie et la  consommation journalière d'un bâtiment référence située à la Réunion pour voir le comportement de notre microréseau en temps réel, dans le Test Drive. Puis, par la validation avec des données réelles du modèle court terme. Les modèles construits avec l'outil Simscape Power Sytems nous permettront de valider les commandes MPPT côté PV et PI côté batterie pour le maintien du bus DC autour d'une valeur référence. Enfin, nous allons finir ce travail par une conclusion et ouvrir des portes de perspective pour d'éventuelles questions soulevées au cours de ce stage.\\ Ainsi, dans le but de se positionner sur l'état de l'énergie sur l'île de la Réunion, une brève étude sur le contexte énergétique à la Réunion et la part des énergies renouvelables à l'échelle humaine sera présentée en annexe.