%% Basierend auf einer TeXnicCenter-Vorlage von Tino Weinkauf.
%%%%%%%%%%%%%%%%%%%%%%%%%%%%%%%%%%%%%%%%%%%%%%%%%%%%%%%%%%%%%%

%%%%%%%%%%%%%%%%%%%%%%%%%%%%%%%%%%%%%%%%%%%%%%%%%%%%%%%%%%%%%
%% HEADER
%%%%%%%%%%%%%%%%%%%%%%%%%%%%%%%%%%%%%%%%%%%%%%%%%%%%%%%%%%%%%
\documentclass[a4paper,twoside,10
pt]{report}

\usepackage{nomencl}
\makenomenclature
%% Deutsche Anpassungen %%%%%%%%%%%%%%%%%%%%%%%%%%%%%%%%%%%%%
%\usepackage[french]{babel}
\usepackage[francais,english]{babel}
\usepackage[T1]{fontenc}
\usepackage[utf8]{inputenc}
\usepackage{appendix}
%\usepackage{french}
%\usepackage[utf8]{inputenc}
\usepackage{lmodern} %Type1-Schriftart f�r nicht-englische Texte
\DeclareTextSymbol{\deg}{T1}{6}
\DeclareTextSymbol{\deg}{OT1}{23}
%\usepackage[french]{babel}
%% Packages f�r Grafiken & Abbildungen %%%%%%%%%%%%%%%%%%%%%%
\usepackage{graphicx} %%Zum Laden von Grafiken
%\usepackage{subfig} %%Teilabbildungen in einer Abbildung
%\usepackage{pst-all} %%PSTricks - nicht verwendbar mit pdfLaTeX
\usepackage{longtable}
%% Beachten Sie:
%% Die Einbindung einer Grafik erfolgt mit \includegraphics{Dateiname}
%% bzw. �ber den Dialog im Einf�gen-Men�.
%%
%% Im Modus "LaTeX => PDF" k�nnen Sie u.a. folgende Grafikformate verwenden:
%%   .jpg  .png  .pdf  .mps
%%
%% In den Modi "LaTeX => DVI", "LaTeX => PS" und "LaTeX => PS => PDF"
%% k�nnen Sie u.a. folgende Grafikformate verwenden:
%%   .eps  .ps  .bmp  .pict  .pntg


%% Packages f�r Formeln %%%%%%%%%%%%%%%%%%%%%%%%%%%%%%%%%%%%%
\usepackage{titlesec}
\titleformat{\chapter}[display]
{\centering\normalfont\huge\bfseries}
{\chaptertitlename\ \thechapter}{15pt}
{\Huge}

\usepackage{amsmath}
\usepackage{amsthm}
\usepackage{amsfonts}
\usepackage{xcolor}
\usepackage[]{hyperref}
\newcommand{\parttoccolor}{}
\newcommand{\chaptertoccolor}{}
\newcommand{\sectiontoccolor}{}
\makeatletter
\renewcommand*\l@part[2]{%
  \ifnum \c@tocdepth >-2\relax
    \addpenalty{-\@highpenalty}%
    \addvspace{2.25em \@plus\p@}%
    \setlength\@tempdima{3em}%
    \begingroup
      \parindent \z@ \rightskip \@pnumwidth
      \parfillskip -\@pnumwidth
      {\leavevmode
       \color{\parttoccolor}\def\@linkcolor{\parttoccolor}%
       \large\bfseries{\mathversion{bold}#1}\hfil
       \hb@xt@\@pnumwidth{\hss
       \def\@linkcolor{\parttoccolor}\color{\parttoccolor}#2}}\par
       \nobreak
         \global\@nobreaktrue
         \everypar{\global\@nobreakfalse\everypar{}}%
    \endgroup
  \fi}
\renewcommand*\l@chapter[2]{%
  \ifnum \c@tocdepth >\m@ne
    \addpenalty{-\@highpenalty}%
    \vskip 1.0em \@plus\p@
    \setlength\@tempdima{1.5em}%
    \begingroup
      \parindent \z@ \rightskip \@pnumwidth
      \parfillskip -\@pnumwidth
      \leavevmode \bfseries
      \advance\leftskip\@tempdima
      \hskip -\leftskip
      \def\@linkcolor{\chaptertoccolor}%
      \color{\chaptertoccolor}{\mathversion{bold}#1}\nobreak\
       \leaders\hbox{$\m@th
        \mkern \@dotsep mu\hbox{.}\mkern \@dotsep
        mu$}\hfil\nobreak\hb@xt@\@pnumwidth{\hss
        \def\@linkcolor{\chaptertoccolor}%
        \color{\chaptertoccolor}#2}\par
      \penalty\@highpenalty
    \endgroup
  \fi}
\def\@dottedtocline#1#2#3#4#5{%
  \ifnum #1>\c@tocdepth \else
    \vskip \z@ \@plus.2\p@
    {\leftskip #2\relax \rightskip \@tocrmarg \parfillskip -\rightskip
     \parindent #2\relax\@afterindenttrue
     \interlinepenalty\@M
     \leavevmode
     \@tempdima #3\relax
     \advance\leftskip \@tempdima \null\nobreak\hskip -\leftskip
     {#4}\nobreak
     \leaders\hbox{$\m@th
        \mkern \@dotsep mu\hbox{.}\mkern \@dotsep
        mu$}\hfill
     \nobreak
     \hb@xt@\@pnumwidth{\hfil\normalfont #5}%
     \par}%
  \fi}
\renewcommand*\l@section{\color{\sectiontoccolor}\def\@linkcolor{\sectiontoccolor}\@dottedtocline{1}{1.5em}{2.3em}}
\def\contentsline#1#2#3#4{%
  \ifx\\#4\\%
    \csname l@#1\endcsname{#2}{#3}%
  \else
      \csname l@#1\endcsname{\hyper@linkstart{link}{#4}{#2}\hyper@linkend}{%
        \hyper@linkstart{link}{#4}{#3}\hyper@linkend
      }%
  \fi
}
\makeatother





%% Zeilenabstand %%%%%%%%%%%%%%%%%%%%%%%%%%%%%%%%%%%%%%%%%%%%
%\usepackage{setspace}
%\singlespacing        %% 1-zeilig (Standard)
%\onehalfspacing       %% 1,5-zeilig
%\doublespacing        %% 2-zeilig


%% Andere Packages %%%%%%%%%%%%%%%%%%%%%%%%%%%%%%%%%%%%%%%%%%
%\usepackage{a4wide} %%Kleinere Seitenr�nder = mehr Text pro Zeile.
%\usepackage{fancyhdr} %%Fancy Kopf- und Fu�zeilen
%\usepackage{longtable} %%F�r Tabellen, die eine Seite �berschreiten
%\usepackage{lastpage}
%\pagestyle{fancy}

%\lhead{Rapport de fin de stage}
%\fancyfoot[LE,RO]{\thepage}/{\pageref{Lastpage}}
%%%%%%%%%%%%%%%%%%%%%%%%%%%%%%%%%%%%%%%%%%%%%%%%%%%%%%%%%%%%%
%% Anmerkungen
%%%%%%%%%%%%%%%%%%%%%%%%%%%%%%%%%%%%%%%%%%%%%%%%%%%%%%%%%%%%%
%
% Zu erledigen:
% 1. Passen Sie die Packages und deren Optionen an (siehe oben).
% 2. Wenn Sie wollen, erstellen Sie eine BibTeX-Datei
%    (z.B. 'literatur.bib').
% 3. Happy TeXing!
%
%%%%%%%%%%%%%%%%%%%%%%%%%%%%%%%%%%%%%%%%%%%%%%%%%%%%%%%%%%%%%


%%%%%%%%%%%%%%%%%%%%%%%%%%%%%%%%%%%%%%%%%%%%%%%%%%%%%%%%%%%%%
%% Optionen / Modifikationen
%%%%%%%%%%%%%%%%%%%%%%%%%%%%%%%%%%%%%%%%%%%%%%%%%%%%%%%%%%%%%

%\input{optionen} %Eine Datei 'optionen.tex' wird hierf�r ben�tigt.
%% ==> TeXnicCenter liefert m�gliche Optionendateien
%% ==> im Vorlagenarchiv mit (Datei | Neu von Vorlage...).
%\usepackage{fancyhdr}
%\pagestyle{fancy}
%\renewcommand\headrulewidth{1pt}
%\fancyhead[L]{\LaTeX \quad C'est super!}
%\fancyhead[R]{Académie de Poitiers}
%\fancyfoot[C]{Rubrique \LaTeX}
%\fancyfoot[R]{\today}

%%%%%%%%%%%%%%%%%%%%%%%%%%%%%%%%%%%%%%%%%%%%%%%%%%%%%%%%%%%%%
%% DOKUMENT
%%%%%%%%%%%%%%%%%%%%%%%%%%%%%%%%%%%%%%%%%%%%%%%%%%%%%%%%%%%%%
\usepackage{multirow}
\usepackage{geometry}
\geometry{top=2cm,bottom=2cm,left=2cm,right=2cm}
\begin{document}
\pagenumbering{roman}
\thispagestyle{empty}
\begin{center}
\textsc{\textbf{\LARGE {Université de la Réunion }}} \vspace{1cm}\\
\includegraphics[scale=0.7]{image.jpg}\vspace{1cm}

%\includegraphics[scale=0.2]{logo1.png}

  
 \textsc{UFR Sciences et Techniques}\vspace{1cm}\\ \Large\textbf{MASTER EN PHYSIQUE ET INGÉNIERIE}\\
 %Master Intérunversitaire d'énergies Renouvelables\vspace{1cm}\\
 \textbf{\underline{Option:}} Conversion des Énergies\vspace{1cm}\\\textsc{\textbf{MÉMOIRE DE FIN D'ÉTUDE} }\vspace{1cm}\\
\rule{0.95\textwidth}{2pt} \\
\LARGE\textbf{Modélisation d'unités de conversion et de stockage d'énergie électrique pour le test en Simulation de stratégies de contrôle d'un système multi sources multi stockages}\\
\rule{0.95\textwidth}{2pt}\vspace{1cm}\\présenté par:\\
\textbf{Abdourahmane THIAM}\vspace{0.5cm}\\Soutenu le 18 juin  2018  
\end{center}
JURY:
\begin{center}
    \begin{tabular}{l l l l}
        Président & Pr Miloud Bessafi & Professeur &Université de la Réunion\\
        Examinateur & Dr Beatrice Morel & Maître de conférence &Université de la Réunion \\
        %Rapporteur& XXX&XXX&Université de la Réunion\\
        Encadreur &Pr Michel Benne & Professeur &Université de la Réunion\\
        Tuteur &Dr Dominique Grondin & Docteur &Chef de projet Gysomate\\
    \end{tabular}
\end{center}
%\begin{center}
%\begin{tabular}
%\texbf{Président} & Pr Sengane MBOJI & Maitre de conférence & UADB\\\hline \vspace{0.5cm}\\\emph\texbf{{Examinateur}} & Dr Biram DIENG & Maitre de conference & UADB \vspace{0.5cm}\\\emph\texbf{{Examinateur}}& Dr Babacar THIAM & Maître de conférence & UADB\vspace{0.5cm}\\\emph\texbf{{Examinateur}}&Dr Mactar FAYE & Maitre de conference & UADB \vspace{0.5cm}\\\emph\texbf{{Encadreur}}&Dr Amadou KANE & Consultant & Ministere de l'Energie\\
%\end{tabular}
%\end{center}

 %%Keine Kopf-/Fusszeilen auf den ersten Seiten.


%% Deckblatt %%%%%%%%%%%%%%%%%%%%%%%%%%%%%%%%%%%%%%%%%%%%%%%%
%% ==> Schreiben Sie hier Ihren Text oder f�gen Sie eine externe Datei ein.

%% Die einfache Version:
%\title{THIAM}
%\author{abda}
%\date{} %%Wenn kommentiert, wird das aktuelle Datum verwendet.
%\maketitle
%\newpage\null\thispagestyle{empty}\newpage
%\pagestyle{plain}
%% Die sch�nere Version:
%\input{deckblatt} %%Eine Datei 'deckblatt.tex' wird hierf�r ben�tigt.
%% ==> TeXnicC



%% Kapitel / Hauptteil des Dokumentes %%%%%%%%%%%%%%%%%%%%%%%
%% ==> Schreiben Sie hier Ihren Text oder f�gen Sie externe Dateien ein.

%\input{intro} %%Eine Datei 'intro.tex' wird hierf�r ben�tigt.

%%%%%%%%%%%%%%%%%%%%%%%%%%%%%%%%%%%%%%%%%%%%%%%%%%%%%%%%%%%%%
%% ==> Im folgenden ein paar Hinweise:
%\input{chap0.tex}
%\listoftables

\input{Dedicace.tex}

\input{Remerciement.tex}
\nomenclature{$T$}{Température en  ($^oK$)}
\nomenclature{$I_r$}{Irradiation solaire ($W/m^2$)}
%\nomenclature{$V{Mpp}$}{Tension de fonctionnement maximale en volt ($V$)}
\nomenclature{$I_l$}{ Courant de l'inductance  (A)}
\nomenclature{$I_{sc}$}{Courant de court circuit (A)}
\nomenclature{$V_{co}$}{Tension de circuit ouvert (V)}
\nomenclature{$V_{Mpp}$}{Tension de fonctionnement maximale des modules PV (V)}
\nomenclature{$SOC$}{État de charge de la batterie  (\%)}
\nomenclature{$SOC_{Max}$}{État de charge maximum des batteries  (\%)}
\nomenclature{$SOC_{Min}$}{État de charge minimum des batteries  (\%)}
%\nomenclature{$Smart Grid$}{ Réseau intelligent}
\nomenclature{$P_{op}$}{ Puissance optimale (W)}
\nomenclature{$P_{Mpp}$}{Puissance au point de fonctionnement maximal (W)}
%\nomenclature{$T$}{Période de hachage en seconde (s)}
\nomenclature{$\alpha$}{Rapport cyclique lié au hacheur dévolteur}
\nomenclature{$f_{PWM}$}{Fréquence de découplage ($H_z$)}
\nomenclature{$N_p$}{Nombre de cellules en parallèles d'un module }
\nomenclature{$N_s$}{Nombre de cellules en séries d'un module}
\nomenclature{$I$}{Courant produit par le module  (A)}
\nomenclature{$I_{ph}$}{Courant des photons  (A)}
\nomenclature{$T_c$}{Température des cellules  ($^oK$)}
\nomenclature{$R_{SH}$}{Résistance en parallèle des cellules ($\Omega$)}
\nomenclature{$R_s$}{Résistance en série des cellules ($\Omega$)}
\nomenclature{$q$}{ Charge élémentaire (eV)}
\nomenclature{$k$}{ Constante de Boltzmann ($j/K$)}
\nomenclature{$A$}{Facteur de qualité}
\nomenclature{$K_I$ }{Coefficient de température ($^oK$)}
\nomenclature{$I_{r}$ }{Irradiation solaire ($W/m^2$)}
\nomenclature{$I_s$}{Courant de saturation de la diode (A)}
\nomenclature{$I_{rs}$}{Courant de saturation inverse (A)}
\nomenclature{$\alpha_{pv}$}{Coefficient de vieillissement des panneaux}
\nomenclature{$V_{pv}$}{Tension des modules phtovoltaique  (V)}
\nomenclature{$P_{pv}$}{Puissance en sortie des modules (W)}
\nomenclature{$E {t}$}{Tension de circuit ouvert de la batterie (V)}
\nomenclature{$E_O$}{Tension de circuit ouvert lorsque la batterie est totalement chargé (V)}
\nomenclature{$C _{Max}$}{Capacité maximale de la batterie en ($Ah$) ou ($Wh$)}
\nomenclature{$R _{bat}$ }{Résistance interne de la batterie  ($\Omega$)}
\nomenclature{$V _{bat}$}{Tension en sortie de la batterie  (V)}
\nomenclature{$DoD_{t}$}{Profondeur de décharge de la batterie (Ah) ici}
\nomenclature{$\Delta_{t}$}{Pas de temps de la simulation (h) ou (s)}
\nomenclature{$I_{bat}$}{Courant en entré de la batterie  (A)}
\nomenclature{$\eta_{bat,disch}$}{ Rendement de décharge de la batterie}
\nomenclature{$$\eta_{bat,ch}$$}{Rendement de charge de la batterie}
\nomenclature{$$P_{bat,ch}(t)$$}{Puissance en charge des batteries (W)}
\nomenclature{$P_{bat,disch}(t)$}{Puissance en décharge des batteries (W)}
\nomenclature{$S_{pv}$}{Surface des panneaux photovoltaique ($m^2$)}
\nomenclature{$D_{pv}$}{Rapport cyclique lié au convertisseur DC/DC entre PV et bus DC}
\nomenclature{$\Delta_{I_{pv}}$}{Ondulation du courant lié au convertisseur DC/DC entre PV et bus DC (A)}
%\nomenclature{$f_{PWM}$}{Fréquence de découplage pour le convertisseur DC/DC entre PV et bus DC  en Hertz}
\nomenclature{$V_{sr}$}{ Tension en sortie de du convertisseur (V)}
\nomenclature{$L_{pv}$}{ Valeur de l'inductance hacheur survolteur pour coté PV  (H)}
\nomenclature{$I_{sr}$}{Courant de sortie au hacheur survolteur  (A)}
\nomenclature{$\Delta_{V_{sr}}$}{Ondulation de tension en sortie  (V)}
\nomenclature{$C_{pv}$}{Capacité de lissage du courant de sortie du hacheur (F)}
%\nomenclature{$MPPT$}{Maximum de power tracking}
%\nomenclature{P $\And$ O}{Perturbation et observation}
\nomenclature{$I_{dc}$}{Courant du bus DC en (A)}
\nomenclature{$I_{charge}$}{Courant consommé par les charges  (A)}
\nomenclature{$V_{dc_{ref}}$}{Tension de référence du bus DC (V)}
\nomenclature{$K_p$}{Gain proportionnel}
\nomenclature{$K_i$}{Gain intégral}
%\nomenclature{$AM$}{Masse d'air atmosphérique}
%\nomenclature{NTIC}{Nouvelle technologie de l'information et de la communication}
%\nomenclature{$PV$}{Photovoltaique}
%\nomenclature{SMA}{Système Multi Agent}
\printnomenclature
\chapter*{Liste des abbréviations}
\chaptermark{Liste des Abréviations}

\renewcommand*{\arraystretch}{1.37}
\begin{longtable}{@{}l @{\hspace{5mm}} l }%p{0.85\linewidth}
SMA                  & Sytème Multi Agent\\
PV                   &Photovoltaique\\
NTIC                 & Nouvelle Technologie de l'Information et de la Communication\\
AM                  & Coefficient de Masse d'air Atmosphérique \\
MPPT              & Maximum Power Point tracking\\
$P\And O$          & Perturbation et observation algorithme\\

\end{longtable}}






\selectlanguage{francais}

\input{ABS.tex}\\
\selectlanguage{english}
\input{Abstrac.tex}
\selectlanguage{francais}
\tableofcontents

%\thispagestyle{headings}
%\thispagestyle{empty}
%\listeofequation
%\thispagestyle{headings}
\listoffigures
\listoftables
\listeofequation
%\thispagestyle{headings}
\thispagestyle{empty}

%\input{abstrac.tex}
%\input{abstrac.tex}
%\input{chap1.tex}

%\thispagestyle{headings}
%\thispagestyle{empty}
%\selectlanguage{français}
%\begin{abstract}
   % ici c'est le résumé en français
%\end{abstract}
%\thispagestyle{empty}
%\input{chap2.tex}
%\selectlanguage{english}
%\begin{abstract}
   % ici c'est le résumé en anglais
%\end{abstract}
%\thispagestyle{empty}
%\begin{list}
    %\item \texbf\Huge{\underline{{Liste des symboles et notations utilisés}}}\\\\
    %\alpha : Coefficient de quelques chose\\
    %\beta : coeff\\\vspace{1cm}
    %\Lambda : coeffie
%\end{list}
    

\pagenumbering{arabic}
\input{Introduction.tex}

%\thispagestyle{headings}

%\input{Context.tex}

\input{chapitre1.tex}
%\thispagestyle{headings}

\input{chapitre2.tex}
%\thispagestyle{headings}



\input{chapitre3.tex}
%\thispagestyle{headings}
%\input{chapitre4.tex}
\input{Conclusion.tex}
%\thispagestyle{headings}
%\input{annexe1.tex}
%\input{annexe2.tex}
\bibliographystyle{plain}
\bibliography{biblio}
\thispagestyle{empty}
\cleardoublepage
\selectlanguage{francais}
\appendix
\input{A1.tex}
\clearpage
\input{A2.tex}
\clearpage
%\input{A3.tex}
\clearpage
\input{A4.tex}
\clearpage
\input{A5.tex}

%\input{annexe1.tex}
%\thispagestyle{empty}
%\input{annexe2.tex}
%\thispagestyle{empty}
\end{document}
%\input{}
